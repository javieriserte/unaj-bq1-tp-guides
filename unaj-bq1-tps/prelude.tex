\usepackage{pifont}             % Maneja Fuentes 
%\usepackage{PTSans}             % Maneja Fuentes Sans Serif
%\usepackage{skak}               % simbolos de ajedrez
\usepackage[T1]{fontenc}        % Texto de alta calidad en pdfs
\usepackage{ae,aecompl}         % Texto de alta calidad en pdfs
\usepackage{pslatex}            % Texto de alta calidad en pdfs
\usepackage{textcomp}           % Cosas de Texto
\usepackage{multicol}           % Para manejar multiples columnas
\usepackage[ansinew]{inputenc}  % Para poner acentos
\usepackage[spanish]{babel}     % Para separar palabras al fin de los renglones.
\usepackage{amsmath}            % Para hacer matemática
\usepackage{amsfonts}           % Para hacer matemática
\usepackage{amssymb}            % Para hacer matemática
\usepackage{enumerate}          % Para Manehar Enumeraciones
\usepackage{vwcol}              % para manejar multiples columas
\usepackage{graphicx}           % Para manejar los gráficos
\usepackage{fancyhdr}           % Para manejar encabezados y pie de páginas
\usepackage{lastpage}           % Crea una referencia a la última página. Util para numerar páginas de la forma "página x de y"
%\usepackage[final]{ps4pdf}      % para incluir archivos eps dentro del documento pdf
\usepackage{array}
\usepackage{soul}
\usepackage{datetime}
\usepackage{natbib}
%\usepackage{boolexpr}


\bibliographystyle{jplainnat}   

% Configuración de página
%%%%%%%%%%%%%%%%%%%%%%%%%
\textwidth 15cm                 % Declaraciones para cambiar el tamaño del área de texto para usar
\textheight 25.7cm              % Declaraciones para cambiar el tamaño del área de texto para usar
\oddsidemargin 0.5cm             % Declaraciones para cambiar el tamaño del área de texto para usar
\evensidemargin 0.5cm            % Declaraciones para cambiar el tamaño del área de texto para usar
\topmargin -1.5cm               % Declaraciones para cambiar el tamaño del área de texto para usar
\headheight 0cm                 % Declaraciones para cambiar el tamaño del área de texto para usar
\headsep 10pt                   % Declaraciones para cambiar el tamaño del área de texto para usar
\marginparwidth 0cm

% Bibliografía
%%%%%%%%%%%%%%%%%%%%%%%%%%%%%%%%%%%%%%%%%%%%%%%%%%%%%%%%%%%%%%%%%%%%%%%%%%%%%%%%%%%%%%%%%%%%%%%%%%%%%
\bibliographystyle{alpha}       % estilo de la bibliografia

% Formato de fecha
%%%%%%%%%%%%%%%%%%%%%%%%%%%%%%%%%%%%%%%%%%%%%%%%%%%%%%%%%%%%%%%%%%%%%%%%%%%%%%%%%%%%%%%%%%%%%%%%%%%%%
\renewcommand{\dateseparator}{-}

% Titulo
%%%%%%%%%%%%%%%%%%%%%%%%%%%%%%%%%%%%%%%%%%%%%%%%%%%%%%%%%%%%%%%%%%%%%%%%%%%%%%%%%%%%%%%%%%%%%%%%%%%%%
\newcommand{\s}{\the\textwidth} 
\newcommand{\materiaName}{Bioqu��mica I}                                   % Guardo el nombre de la asignatura.

\newcommand{\fecha}{\ddmmyyyydate\today}                                    % Guardo la fecha.
\newcommand{\titulo}{                                                     % Para hacer el Título como me gusta
  \begin{center}                                                          % Para hacer el Título como me gusta
  \vspace*{-22pt}\large\textbf{\materiaName}\\                            % Para hacer el Título como me gusta        
  \vspace{2pt}\normalsize{\seminarioName}\\                               % Para hacer el Título como me gusta
  \vspace{2pt}\small{\textbf{U}niversidad \textbf{N}acional \textbf{A}rturo \textbf{J}auretche}\\                                                 % Para hacer el Título como me gusta
  \vspace{-6pt}\rule{\textwidth}{0.4pt}
  \end{center}                                                            % Para hacer el Título como me gusta
}

% Encabezado y Pie de Página
%%%%%%%%%%%%%%%%%%%%%%%%%%%%%%%%%%%%%%%%%%%%%%%%%%%%%%%%%%%%%%%%%%%%%%%%%%%%%%%%%%%%%%%%%%%%%%%%%%%%%
\headheight=14pt                                                          % Creo el encabezado y pie página que quiero.
\renewcommand{\headrulewidth}{0.4pt}                                      % Creo el encabezado y pie página que quiero.
\renewcommand{\footrulewidth}{0.4pt}                                      % Creo el encabezado y pie página que quiero.
\pagestyle{fancy}%                                                        % Creo el encabezado y pie página que quiero.
\fancyfoot{}                                                              % Creo el encabezado y pie página que quiero.
\lfoot{\footnotesize{P�gina \thepage\ de \pageref{LastPage}}}             % Creo el encabezado y pie página que quiero.
\lhead{\footnotesize{\materiaName}}                                       % Creo el encabezado y pie página que quiero.
\rhead{\footnotesize{\fecha\  - \seminarioName}}                          % Creo el encabezado y pie página que quiero.

% Fuentes
%%%%%%%%%%%%%%%%%%%%%%%%%%%%%%%%%%%%%%%%%%%%%%%%%%%%%%%%%%%%%%%%%%%%%%%%%%%%%%%%%%%%%%%%%%%%%%%%%%%%%
\renewcommand{\familydefault}{\sfdefault}                                 % Cambia todas las fuentes a sans Serif

% Enumeraciones
%%%%%%%%%%%%%%%%%%%%%%%%%%%%%%%%%%%%%%%%%%%%%%%%%%%%%%%%%%%%%%%%%%%%%%%%%%%%%%%%%%%%%%%%%%%%%%%%%%%%%
\renewcommand{\theenumi}{\alph{enumi}}
\renewcommand{\labelenumi}{\textbf{\theenumi}}
\renewcommand{\theenumii}{\arabic{enumii}}
\renewcommand{\labelenumii}{\theenumii}